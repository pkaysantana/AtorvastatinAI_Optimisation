\documentclass[fleqn,10pt]{olplainarticle}
% Use option lineno for line numbers 

\title{Optimising Atorvastatin Manufacturing Through AI-Driven Predictive Maintenance and Process Design: A Chemical Engineering Perspective}


\author[1]{20575412}

\keywords{Define acronyms (e.g., GMP, CQA) and variables used frequently in the report.}

\begin{abstract}
	•	Maximum 200 words summarising:
	•	Objectives (e.g., integrating predictive maintenance and process optimisation).
	•	Key methodologies (e.g., IoT sensor data analysis, AspenTech simulations).
	•	Expected outcomes (e.g., cost savings, reduced downtime).
	•	Use precise technical language to highlight the significance of your work.
\end{abstract}

\begin{document}

\flushbottom
\maketitle
\thispagestyle{empty}



	

\section*{Introduction}

\subsection{Background}
\subsection{Problem Statement}
\subsection{Objectives}
\section{Literature Review}
\subsection{AI in Pharmaceutical Manufacturing}
\subsection{Chemical Engineering Context}
\subsection{Case Studies}
\section{Methodology}
\subsection{Predictive Maintenance Framework}
\subsubsection{Sensor Configuration}
\subsubsection{Data Analysis}
\subsection{Process Optimisation Workflow}
\subsubsection{Integration with APC}
\subsection{Validation Strategy}
\subsubsection{Regulatory Compliance}
\subsubsection{Experimental Testing}
\section{Results \& Discussion}
\subsection{Predictive Maintenance Outcomes}
\subsection{Process Optimisation Results}
\subsection{Discussion Points}
\section{Conclusion \& Recommendations}
\subsection{Summary}
\subsection{Recommendations}
\subsubsection{Scale-up predictive maintenance across production lines to maximise impact}

\section*{Acknowledgments}


\bibliography{sample}

\end{document}